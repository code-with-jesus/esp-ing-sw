\part{Methods}
\chapter{Spectral Graph Theory}

When we talk about graph theory, maybe we remember  data structures like trees and related topics such as find the shortest path, breadth-first search or depth-first search, but when we said \emph{spectral} probably we are not clear about what it refers. In short, Spectral Graph Theory studies the properties of a special graph called \emph{Laplacian} through its eigenvalues and eigenvectors, however, there are much more than this.

Riemannian manifold
Laplace Beltrami Operator
Dynamic Laplacian

Assumption that the data lies in a lower dimensional manifold in a high dimensional space

Eigenfunction of the Laplacian is very usefull in the analysis on Riemannian manifold. It is the key to carry an analog of Fourier series on manifolds 

we may interpret the eigenfunction $\varphi$ as the probability density of a quantum particle in the energy state $\lambda$. That is, the probability that a particle in the state $\varphi$ belong to the set A

a version of the dynamic isoperimetric problem, generalised to the situation where the dynamics need not be volume preserving
A manifold M is splited by a compact hypersurface $\Gamma$ in two submanifolds M\textsubscript{1} and M\textsubscript{2}. Then is necesary transform each manifold M\textsubscript{1} and M\textsubscript{2} to N\textsubscript{1} and N\textsubscript{2}  respectively

the initial manifold M(0) is transformed under the smooth flow maps 




\textit{
	``Descriptors are some set of numbers that are produced to describe a given shape. The shape may not be entirely reconstructable from the descriptors, but the descriptors for different shapes should be different enough that the shapes can be discriminated"
}
