\part{Preliminaries}
\chapter{Introduction}

In the last years clustering data stream ...


\section{Problem Statement}

\subsection{Challenges}


\subsection{Thesis Goals}
\subsubsection{Main Goal}
To propose a model to identify clusters of a data stream based on spectral graph theory. The following are the general purposes of this research:

\subsubsection{Specific Objetives}

\begin{itemize}
	\item To propose an approach of data stream clustering based on spectral graph theory.
	\item To design a model to manage the evolution on the clustering result of previous approach.
	\item To implement the model and to evaluate its effectiveness and efficiency.
\end{itemize}

\section{Thesis Scope}

The scope of this work


\section{Main contributions}
The following is the outline of the main contributions of this work:


\section{Outline}
The remainder of this document is organized as follows. Next chapter gives a review of how data stream clustering has been envolved. Chapter 3 shows  an overall background of how Spectral Graph Theory can be applied as a factible solution in the clustering problem. In Chapter 4 a semi-supervised learning framework based on direct eigenspace mapping is presented. Chapter 5 shows how to use the previous learning framework to handle datastreams. Chapter 6 presents the evaluations of the proposed model. Chapter 7 shows a summary that discusses the main aspects of this research, presents the conclusions and the future work.
